\documentclass[UTF8]{ctexart} % to support Chinese

\usepackage{amsmath}    % math
% \usepackage{amssymb}
\usepackage{listings}   % 代码段

\title{Algorithms Homework 3}
\author{PB22111620 Ai Chang}

\begin{document}
\begin{sloppypar}   % 避免边界越界

\maketitle

\section*{Q1}
\begin{enumerate}
    \item 插入排序、冒泡排序、希尔排序、快速排序是稳定的,堆排序不是稳定的。
    \item 对每个元素添加一个属性 `position' 储存其初始位置信息,在排序开始前按顺序
    赋值1, 2, \dots, n, 排序完成后从前到后遍历,如果遇到关键值相同的子串,对该字串
    以 `position' 为关键值再次进行排序。该方法带来额外的时间开销为O(n), 空间开销为O(n)。
\end{enumerate}

\section*{Q2}
    修改后的代码如下:
    \begin{lstlisting}[language=C]
    int stack[N];

    void push(p, r);
    void pull(*p, *r);

    Quicksort(A, p, r){
        if(p < r) {
            q = Partition(A, p, r);
            push(p, q - 1);
            push(q + 1, r);
            while(stack is not empty){
                pull(&p, &r);
                Quicksort(A, p, r);
            }
        }
    }
    \end{lstlisting}

\section*{Q3}
易知,对于一棵二叉搜索树,每个节点需要进行比较的次数为该节点到根节点的距离。因此,需要考虑
构造出的二叉搜索树所有节点到根节点的距离的和最小的情况。显然,这种情况要求二叉树较为均衡,
其节点到根节点距离的和等价于一棵完全二叉树。

设该完全二叉树 T 共有 n 个节点,树高为 h, 其中第 h 层有 k 个节点,显然有
$$2^0 + 2^1 + \dots + 2^{h-2} + 1 \leq n \leq 2^0 + 2^1 + \dots + 2^{h-1}$$
即$$\log(n+1) \leq h \leq \log n + 1$$
以及$$1 \leq k \leq 2^{h-1}$$
设所有节点到根节点的距离的和为 SUM, 则$$SUM = 2^0*0 + 2^1*1 + \dots + 2^{h-2}*(h-2) + k*(h-1)$$
即$$SUM = \Sigma_{i=1}^{h-2}i*2^i + (h-1)*k$$
而$$\Sigma_{i=1}^n i*2^i = (n-1)2^{n+1}+2$$
故$$(h-1)*2^{h-1}+h+1 \leq SUM \leq (h-2)*2^h+2$$
即$$\frac{1}{2}(n+1)\log(n+1)+\log(n+1)-\frac{n+1}{2}+1 \leq SUM \leq 2n\log n-2n+2$$
显然$$SUM(n) = \Omega(n\log n)$$命题得证。

\section*{Q4}
设 $X_n$ 为 n 个节点的二叉搜索树的树高,$Y_n = 2^{X_n}$。
若根节点为该序列中第 i 小的数,则根节点的左子树共有 $i-1$
个节点,右子树有 $n-i$ 个节点。则有
\begin{equation*}
    Y_n = 2 \times max\{Y_{i-1}, Y_{n-i\}}
\end{equation*}
因为输入是随机的,所以抽到第 i 小的节点的概率为
\begin{equation*}
    Prob(i) = \frac{1}{n}, \forall i = 1, \dots, n
\end{equation*}
故有
\begin{align*}
    E(Y_n) &= E(2 \times max\{Y_{i-1}, Y_{n-i}\})\\
           &= 2\Sigma_{i=1}^{n} Prob(i)E(max\{Y_{i-1}, Y_{n-i}\})\\
           &\leq \frac{2}{n} \Sigma_{i=1}^{n}(E(Y_{i-1})+E(Y_{n-i}))\\
           &= \frac{4}{n} \Sigma_{i=0}^{n-1}E(Y_i)
\end{align*}
由数学归纳法,易证 $$E(Y_n) \leq \frac{1}{4} C_{n+3}^{3}$$
由$$Y_n = 2^{X_n}$$
即$$E(Y_n) = E(2^{X_n})$$
可得$$2^{E(X_n)} \leq E(Y_n) \leq \frac{1}{4}C_{n+3}^3$$
而$$C_{n+3}^3 = \frac{(n+3)!}{3!n!} = \frac{(n+3)(n+2)(n+1)}{6}~O(n^3)$$
故$$2^{E(X_n)} \leq O(n^3)$$
所以$$E(X_n) \leq O(\log n^3) = O(\log n)$$

类似的,我们同样可以得到
\begin{align*}
    E(Y_n)  &= 2\Sigma_{i=1}^{n} Prob(i)E(max\{Y_{i-1}, Y_{n-i}\})\\
            &\geq \frac{2}{n} \frac{E(Y_{i-1})+E(Y_{n-i})}{2}\\
            &= \frac{2}{n} \Sigma_{i=0}^{n-1} E(Y_i)
\end{align*}
后续步骤与前面的证明同理,最终可得$$E(X_n) \geq \Omega(\log n)$$
综上,$$E(X_n) = \Theta(\log n)$$
所以以随机的输入构建的二叉搜索树的平均节点深度的期望为$\Theta(\log(n))$,命题得证

\section*{Q5}
可以通过修改二叉树节点的数据结构,来实现该要求,代码如下:
    \begin{lstlisting}[language=C]
    typedef struct Node{
        int key, order;
        struct Node *lchild, *rchild;
    }Node;

    Node* Querykth(*root, k){
        if(root->order < k){
            Querykth(root->lchild, k);
        }
        else if(root->order = k){
            return root;
        }
        else{
            Querykth(root->rchild, k);
        }
    }
    \end{lstlisting}
易知,设目标节点的深度为 k, 则共需要 $2*(k-1)$ 次比较,所以该操作时间复杂度为$O(h)$
\end{sloppypar}
\end{document}