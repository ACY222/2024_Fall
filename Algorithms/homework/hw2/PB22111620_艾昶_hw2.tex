\documentclass{article}
\usepackage{amsmath}    % some mathematical symbols
\usepackage{amssymb}
\usepackage{listings}   % lists


\title{Algorithms Homework 2}
\author{PB22111620  Ai Chang}

\begin{document}
\begin{sloppypar}

\maketitle

\section*{Q1}
\begin{enumerate}
    \item $\because T(n) = 4T(\frac{n}{2}) + n^2 \lg^2 n$

    $\therefore a = 4, b = 2,$
    $f(n) = n^2 \lg^2 n = \Theta(n^2 \lg^2 n) = \Theta(n^{log_b^a}\lg^kn)$

    According to Master Method, $T(n) = \Theta(n^{log_b^a}\lg^{k+1}n) = \Theta(n^2\lg^3n)$

    \item $\because T(n) = T(2n/3) + 2^{\lceil \lg n \rceil}$
    $\therefore a = 1, b = \frac{3}{2},$

    $f(n) = 2^{\lceil \lg n \rceil} = 2^{\lg n + (\lceil \lg n \rceil - \lg n)} = n2^\alpha, \alpha = \lceil \lg n \rceil - \lg n \in [0, 1)$

    $\therefore \log_b^a = \log_{\frac{3}{2}}^{1} = 0, f(n) = \Theta(n) = \Theta(n^{log_b^a + 1})$

    According to Master Method, $T(n) = \Theta(f(n)) = \Theta(n)$
\end{enumerate}

\section*{Q2}
\begin{enumerate}
    \item $\because T(n) = 2T(n/4) + 3T(n/6) + n\lg n$
    \begin{align*} \therefore
        T(24)   &= 2T(6) + 3T(4) + 24\lg 24\\
                &= 2(2T(3/2) + 3T(1) + 6\lg 6) + 3(2T(1) + 3T(2/3) + 4\lg 4) + 24(3 + \lg 3)\\
                &= 2(2(2T(3/8) + 3T(1/4) + \frac{3}{2}\lg{\frac{3}{2}}) + 3T(1) + 6(1 + \lg 3))\\
                &\quad + 3(2T(1) + 3T(2/3) + 4\lg 4) + 24(3 + \lg 3)
    \end{align*}
    $\because T(x) = 1, \text{when } x \leq 1$

    $\therefore T(24) = 143 + 42\lg 3 \approx 209.57$

    \item My program is as follow:
        \begin{lstlisting}[language=C]
            #include<stdio.h>
            #include<math.h>

            double T(double n){
                if(n <= 1){
                    return 1;
                }
                else{
                    return 2*T(n/4)+3*T(n/6)+n*log2(n);
                }
            }
            int main(){
                double input;
                scanf("%lf", &input);
                printf("%lf\n", T(input));
                return 0;
            }
        \end{lstlisting}
    According to the program,
    \begin{align*}
        T(100) &\approx 1616.67\\
        T(1000) &\approx 29728.49\\
        T(10000) &\approx 481351.65&
    \end{align*}
    I guess that $T(n) = \Theta(n\lg^2n)$
    \item By drawing the recursion tree, we know that(I haven't learnt how to draw in LaTeX)
        \begin{enumerate}
            \item The 1st layer, $T(n) = 2T(n/4) + 3T(n/6) + n\lg n, \therefore t_1 = n\lg n$
            \item The 2st layer,
                \begin{align*}
                    2T(n/4) + 3T(n/6) &= 2(2T(n/16) + 3T(n/24) + n/4\lg(n/4))\\
                        &\quad +3(2T(n/24)+3T(n/36)+n/6\lg(n/6)) \\
                        &= 4T(n/16) + 12T(n/24) + n(\lg n - \lg(24)/2)\\
                        &= 4T(n/16) + 12T(n/24) + n\lg n, n \to \infty
                \end{align*}

                $\therefore t_2 = n\lg n$
            \item \ldots
        \end{enumerate}
        We find that the time of each layer is $n\lg n$. And the number of layers, obviously, is $\lg n$.

        $\therefore T(n) = \Theta(n\lg^2n)$
    \item According to Akra-Bazzi principle and the given equation, we know that
        $a_1 = 2, b_1 = 1/4, a_2 = 3, b_2 = 1/6, \sum_{i = 1}^{k} a_ib_i^p = 1$
        $\therefore p = 1$
        \begin{align*}
            T(n) &= \Theta(n^1(1+\int_{1}^{n}\frac{x\lg x}{x^2} \rm dx))\\
                 &= \Theta(n(1+\frac{\ln 2}{2}\lg^2n))\\
                 &= \Theta(n\lg^2n)&
        \end{align*}
\end{enumerate}

\section*{Q3}
\begin{lstlisting}[language=C]
    NEW_MAX_HEAPIFY(A, i){
        int stack[A.heap-size] = {0};
        push(i);
        while(stack is not empty){
            pop(root);
            l = LEFT(root);
            r = RIGHT(root);
            if(l <= A.heap-size and A[l] > A[root]){
                largest = l;
            }
            else{
                largest = root;
            }
            if(r <= A.heap-size and A[r] > A[largest]){
                largest = r;
            }
            if(largest != root){
                exchange A[root] with A[largest];
                push(largest);
            }
        }
    }
\end{lstlisting}

\section*{Q4}
Set a random variable x to represent the ratio of the smaller part, then x is also the pivot's ``distance'' to the nearer edge.

Obviously, $x ~ U(0, \frac{1}{2}), \text{namely }f(x) = 2I_{(0, \frac{1}{2})}(x)$

$\therefore P(\text{A more balanced division}) = P(\alpha < x < \frac{1}{2}) = \int_{\alpha}^{1/2}f(x)\rm dx = 1 - 2\alpha$

\section*{Q5}
\begin{enumerate}
    \item Based on the leftmost element
        \begin{enumerate}
            \item Take 1 as the pivot, scan the list from left to right, and 
                    move the elements smaller than pivot to the left side.
                    Elements 2, 3, \dots, 7 are all larger than 1 therefore do
                    not move. Finally put 1 in the approriate position and we get
                    [1,3,3,5,6,7,2]. Then recursively sort the left part and right part.
            \item The right part is empty, over.
            \item Sort the right part: take 2 as the pivot. Similarly to step (a),
                    we finally get [2,3,3,6,7,5]. Then recursively sort the left part
                    and the right part.
            \item Finally, we get the sorted list.
        \end{enumerate}
    \item Based on the median of the three specified numbers
        \begin{enumerate}
            \item Take 2(the median of [1,2,5]) as the pivot, scan the list
                    from left to right\ldots Finally we get [1,2,3,3,5,6,7].
            \item The left part has only one element, so we don't need to sort it.
                    Take 5 as the pivot to sort the right part. Recursively repeat
                    these steps\dots
            \item Finally, we get the sorted list.
        \end{enumerate}
    \item Select the most balanced partition.
            \begin{enumerate}
                \item Take 3 as pivot, and then we get [2,3,1] as the left part
                        and [6,7,5] as the right part.
                \item Then take 2 as pivot to sort the left part, and then we get
                        [1,2,3]. Both parts' length is 1, over. And take 6 as pivot
                        to sort the right part, then we get [5,6,7]. It's over, too.

            \end{enumerate}
\end{enumerate}
\end{sloppypar}
\end{document}
